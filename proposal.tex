% !TEX program = pdflatex
\documentclass[12pt,english]{article}


\usepackage{lmodern}
\renewcommand{\sfdefault}{lmss}
\renewcommand{\ttdefault}{lmtt}
\usepackage[T1]{fontenc}
\usepackage[utf8]{inputenc}
\usepackage{listings}
\usepackage{geometry}
\geometry{verbose,tmargin=1in,bmargin=1in,lmargin=1in,rmargin=1in}
\usepackage{url}
\usepackage{graphicx}
\usepackage{setspace}
\usepackage[authoryear]{natbib}

\doublespacing

\makeatletter

\title{\textbf{rtweet: Collecting Twitter Data}}
\author{A proposal for \\\emph{POLS 904: Statistical Computing} \\ by Michael W. Kearney}
\date{Fall 2016}

\usepackage{color}

\definecolor{red}{rgb}{.8, .2, .2}
\definecolor{blue}{rgb}{.2, .3, .8}
\definecolor{gray}{RGB}{245,255,250}

\lstset{
    tabsize=2,
    language=R,
    basicstyle=\ttfamily,
    aboveskip=0in,
    belowskip=0in,
    basicstyle=\small\ttfamily,
    keywordstyle=\color{blue},
    breaklines=true,
    otherkeywords={<-, paste0, POST, config, GET, fromJSON, http_error},
    stringstyle=\color{red},
    numberstyle=\tiny\defaultcolor,
    numbersep=6pt,
    basewidth=0.3em,
    fontadjust=true,
    flexiblecolumns=true,
    keepspaces=false,
    showspaces=false,
    lineskip=-.5pt,
    xleftmargin=.25in,
    xrightmargin=.25in
}

\usepackage{hyperref}
\hypersetup{
     colorlinks   = true,
     citecolor    = blue,
     linkcolor    = black,
     urlcolor     = blue
}

\usepackage{fancyvrb}
\newcommand{\VerbBar}{|}
\newcommand{\VERB}{\Verb[commandchars=\\\{\}]}
\DefineVerbatimEnvironment{Highlighting}{Verbatim}{commandchars=\\\{\}}
% Add ',fontsize=\small' for more characters per line
\usepackage{framed}
\definecolor{shadecolor}{RGB}{248,248,248}
\newenvironment{Shaded}{\begin{snugshade}}{\end{snugshade}}
\newcommand{\KeywordTok}[1]{\textcolor[rgb]{0.13,0.29,0.53}{\textbf{{#1}}}}
\newcommand{\DataTypeTok}[1]{\textcolor[rgb]{0.13,0.29,0.53}{{#1}}}
\newcommand{\DecValTok}[1]{\textcolor[rgb]{0.00,0.00,0.81}{{#1}}}
\newcommand{\BaseNTok}[1]{\textcolor[rgb]{0.00,0.00,0.81}{{#1}}}
\newcommand{\FloatTok}[1]{\textcolor[rgb]{0.00,0.00,0.81}{{#1}}}
\newcommand{\ConstantTok}[1]{\textcolor[rgb]{0.00,0.00,0.00}{{#1}}}
\newcommand{\CharTok}[1]{\textcolor[rgb]{0.31,0.60,0.02}{{#1}}}
\newcommand{\SpecialCharTok}[1]{\textcolor[rgb]{0.00,0.00,0.00}{{#1}}}
\newcommand{\StringTok}[1]{\textcolor[rgb]{0.31,0.60,0.02}{{#1}}}
\newcommand{\VerbatimStringTok}[1]{\textcolor[rgb]{0.31,0.60,0.02}{{#1}}}
\newcommand{\SpecialStringTok}[1]{\textcolor[rgb]{0.31,0.60,0.02}{{#1}}}
\newcommand{\ImportTok}[1]{{#1}}
\newcommand{\CommentTok}[1]{\textcolor[rgb]{0.56,0.35,0.01}{\textit{{#1}}}}
\newcommand{\DocumentationTok}[1]{\textcolor[rgb]{0.56,0.35,0.01}{\textbf{\textit{{#1}}}}}
\newcommand{\AnnotationTok}[1]{\textcolor[rgb]{0.56,0.35,0.01}{\textbf{\textit{{#1}}}}}
\newcommand{\CommentVarTok}[1]{\textcolor[rgb]{0.56,0.35,0.01}{\textbf{\textit{{#1}}}}}
\newcommand{\OtherTok}[1]{\textcolor[rgb]{0.56,0.35,0.01}{{#1}}}
\newcommand{\FunctionTok}[1]{\textcolor[rgb]{0.00,0.00,0.00}{{#1}}}
\newcommand{\VariableTok}[1]{\textcolor[rgb]{0.00,0.00,0.00}{{#1}}}
\newcommand{\ControlFlowTok}[1]{\textcolor[rgb]{0.13,0.29,0.53}{\textbf{{#1}}}}
\newcommand{\OperatorTok}[1]{\textcolor[rgb]{0.81,0.36,0.00}{\textbf{{#1}}}}
\newcommand{\BuiltInTok}[1]{{#1}}
\newcommand{\ExtensionTok}[1]{{#1}}
\newcommand{\PreprocessorTok}[1]{\textcolor[rgb]{0.56,0.35,0.01}{\textit{{#1}}}}
\newcommand{\AttributeTok}[1]{\textcolor[rgb]{0.77,0.63,0.00}{{#1}}}
\newcommand{\RegionMarkerTok}[1]{{#1}}
\newcommand{\InformationTok}[1]{\textcolor[rgb]{0.56,0.35,0.01}{\textbf{\textit{{#1}}}}}
\newcommand{\WarningTok}[1]{\textcolor[rgb]{0.56,0.35,0.01}{\textbf{\textit{{#1}}}}}
\newcommand{\AlertTok}[1]{\textcolor[rgb]{0.94,0.16,0.16}{{#1}}}
\newcommand{\ErrorTok}[1]{\textcolor[rgb]{0.64,0.00,0.00}{\textbf{{#1}}}}
\newcommand{\NormalTok}[1]{{#1}}

\begin{document}

\maketitle

\section{Background}

My research examines the intersection of new media environments and
political communication using innovative quantitative research methods.
My current work analyzes ``big data'' to better understand the
relationship between selective exposure and social media. In working to systematically collect and wrangle large amounts of data,
I created and now maintain \emph{rtweet}, the successor package to
\emph{twitteR} for collecting Twitter data. Originally developed to collect
data for my dissertation, \emph{rtweet} now features numerous diverse
methods for interacting with Twitter APIs and represents the future of
open-source efforts in collecting Twitter data.

\section{Proposal}

Most of the work toward creating the package is done. I have
already published \textit{rtweet} on CRAN and have received a fair amount
of feedback via email and Github. I still want to add a couple new
features and implement at least a handful of improvements, but my
next big goal is to advertise the package and have my work count toward
something a little more tangible. With this in mind, for my semester project
in \textit{Statistical Computing and Foundations}, I propose I write and
submit an article about \textit{rtweet} to the \textit{R Journal}.

\section{Details}

As an introductory piece, the article will provide three different overviews.
First, the article will provide an introduction to working with
Twitter’s APIs. This section will also include some basic discussion about the
the nature and limitations of publically available Twitter data. Second,
I will provide an overview of the major functions and features in
\textit{rtweet}. Third, I will provide quick examples of different statitical
analyses, including sentiment [textual] analysis, social network analysis,
and time series.

The R Journal encourages authors to discuss similar packages, so I would like
to provide some points of comparison between \textit{rtweet} and \textit{twitteR}. In the third section described above, I would also demonstrate
how to make \textit{rtweet} compatible with several statistical packages.
This may also feature integration with \textit{data.table} or other ways
to work with relational databases.

\end{document}
