% !TeX root = RJwrapper.tex
\title{rtweet: Collecting Twitter Data}
\author{by Michael W. Kearney}

\maketitle

\abstract{
An implementation of calls designed to extract and organize Twitter data via Twitter's REST and stream API's. Functions formulate GET and POST requests and convert response objects to more user friendly structures, e.g., data frames or lists. Specific consideration is given to functions designed to return tweets, friends, and followers.
}

Introductory section which may include references in parentheses
\citep{R}, or cite a reference such as \citet{R} in the text.

\section{Working with APIs}

This section may contain a figure such as Figure~\ref{figure:rlogo}.

\begin{figure}[htbp]
  \centering
  \includegraphics{Rlogo}
  \caption{The logo of R.}
  \label{figure:rlogo}
\end{figure}

\section{rtweet Package}

My research examines the intersection of new media environments and
political communication using innovative quantitative research methods.
My current work analyzes ``big data'' to better understand the
relationship between selective exposure and social media.

In working to systematically collect and wrangle large amounts of data,
I created and now maintain \emph{rtweet}, the successor package to
\emph{twitteR} for collecting Twitter data. Originally developed to collect
data for my dissertation, \emph{rtweet} now features numerous diverse
methods for interacting with Twitter APIs and represents the future of
open-source efforts in collecting Twitter data.

\begin{example}
  library(rtweet)
  tw <- search_tweets(``rstats'')
  head(tw)
\end{example}

\section{proposal}

Most of the work toward creating the package is already done. I have
already published \textit{rtweet} on CRAN and have received a fair amount
of feedback via email and Github. I still want to add a couple new
features and implement at least a handful of improvements, but my
next big goal is to advertise the package and have my work count toward
something a little more tangible. With this in mind, for my course project
in \textit{Statistical Computing and Foundations}, I propose that I write and
submit an article about \textit{rtweet} to the \textit{R Journal}.

\bibliography{RJreferences}

\address{Michael W. Kearney\\
  University of Kansas\\
  1440 Jayhawk Blvd, Bailey Rm. 102\\
  Lawrence, KS 66044\\}
\email{email@mikewk.com}
